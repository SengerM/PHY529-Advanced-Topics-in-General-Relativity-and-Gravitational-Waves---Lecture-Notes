%% LyX 2.3.6.1 created this file.  For more info, see http://www.lyx.org/.
%% Do not edit unless you really know what you are doing.
\documentclass[english]{article}
\usepackage[T1]{fontenc}
\usepackage[latin9]{inputenc}
\usepackage[a4paper]{geometry}
\geometry{verbose,tmargin=2cm,bmargin=2cm,lmargin=1cm,rmargin=1cm}
\usepackage{fancyhdr}
\pagestyle{fancy}
\usepackage{color}
\usepackage{babel}
\usepackage{float}
\usepackage{units}
\usepackage{textcomp}
\usepackage{mathrsfs}
\usepackage{url}
\usepackage{amsmath}
\usepackage{amssymb}
\usepackage{esint}
\PassOptionsToPackage{normalem}{ulem}
\usepackage{ulem}
\usepackage[unicode=true,pdfusetitle,
 bookmarks=true,bookmarksnumbered=true,bookmarksopen=false,
 breaklinks=false,pdfborder={0 0 1},backref=false,colorlinks=false]
 {hyperref}

\makeatletter
%%%%%%%%%%%%%%%%%%%%%%%%%%%%%% User specified LaTeX commands.
\usepackage{/home/alf/cloud/lib/lyx/my_preamble}

\makeatother

\usepackage{listings}
\lstset{keywordstyle={\color{keyword_color}\ttfamily\bfseries},
commentstyle={\color{comentarios_color}\itshape},
emphstyle={\color{red}},
breaklines=true,
basicstyle={\ttfamily},
stringstyle={\color{cadenas_color}},
identifierstyle={\color{identifier_color}},
backgroundcolor={\color{fondocodigo_color}},
keepspaces=true,
numbers=left,
xleftmargin=2em,
frame=leftline,
rulecolor={\color{black}},
numbersep=5pt,
tabsize=3}
\begin{document}
\input{\string~/cloud/lib/lyx/macros2020.tex}

\global\long\def\COVARIANTINDEX#1{\hspace{0cm}{}_{#1}}%

\global\long\def\CONTRAVARIANTINDEX#1{\hspace{0cm}{}^{#1}}%

\global\long\def\sun{\odot}%

\global\long\def\TACHADO#1{\underline{#1}}%

\global\long\def\SNR{\text{SNR}}%

\title{PHY529 Advanced Topics in General Relativity and Gravitational Waves
- Lecture Notes}
\author{Mat�as Senger}
\date{\today}

\maketitle
Course website: \url{https://www.physik.uzh.ch/en/teaching/PHY529/HS2021.html}.

\tableofcontents{}
\begin{thebibliography}{1}
\bibitem{Reference: Saulson 2013 gravitational wave detection principles and practice}Peter
R. Saulson, Gravitational wave detection: Principles and practice,
Comptes Rendus Physique, 2013, \url{https://doi.org/10.1016/j.crhy.2013.01.007}.

\bibitem{Reference: Jolien Gravitational Wave Physics and Astronomy BOOK}Gravitational-Wave
Physics and Astronomy, Jolien D.E. Creighton and W. C. Anderson.
\end{thebibliography}

\section*{2021-Sept-23}

Read this for next week \url{https://www.sciencedirect.com/science/article/pii/S163107051300008X}.

Nice summary of gravitational waves: Gravitational-Wave Physics and
Astronomy, Jolien D.E. Creighton and W. C. Anderson.

\section{Summary of the paper \textquotedblleft Gravitational wave detection:
Principles and practice\textquotedblright}

Here I am summarizing the paper we had to read, \cite{Reference: Saulson 2013 gravitational wave detection principles and practice}.

\subsection{What is a gravitational wave}

A gravitational wave is a solution to Einstein's equation in the weak
field limit\footnote{Do we need the weak field limit? In this limit the waves are harmonic,
are there ``non-linear waves'' in the full theory?}. Gravitational waves are \textbf{transverse traceless waves traveling
at the speed of light}. 
\begin{itemize}
\item Transverse: We all know what it means, the same as an electromagnetic
field. The perturbation that the wave produces is transverse with
its direction of propagation.
\item Traceless: The waves have equal and opposite effects in two perpendicular
directions.
\end{itemize}
A wave traveling in the $\VERSOR z$ direction produces relative motion
in the $\VERSOR x$ and $\VERSOR y$ directions of free masses. The
motion in $\VERSOR x$ and $\VERSOR y$ is identical (thus traceless
waves). This are physical motions that can e.g. stretch a spring.

A gravitational wave traveling through flat space has a metric 
\begin{equation}
g_{\mu\nu}=\eta_{\mu\nu}+h_{\mu\nu}\label{Equation: metric with perturbation from a wave}
\end{equation}
 where $\eta$ is the Minkowski metric and $h$ a small perturbation.
If the wave travels in the $\VERSOR z$ direction then 
\[
h_{\mu\nu}\sim\SQBRACKETS{\begin{matrix}0\\
 & a & b\\
 & b & -a\\
 &  &  & 0
\end{matrix}}
\]
 where $a$ and $b$ represent the two possible polarizations of the
wave. I.e. 
\[
h=ah_{+}+bh_{\times}
\]
 where $h_{+}\sim\SQBRACKETS{\begin{matrix}0\\
 & 1\\
 &  & -1\\
 &  &  & 0
\end{matrix}}$ and $h_{\times}\sim\SQBRACKETS{\begin{matrix}0\\
 & 0 & 1\\
 & 1 & 0\\
 &  &  & 0
\end{matrix}}$ are the ``basis tensors'' representing each polarization, called
``plus'' and ``cross'' respectively. In \ref{Figure: polarization of a gravitational wave}
the two polarization modes are represented graphically. 

\begin{figure}[H]
\htmltag{tag_name=image}{src=https://i.stack.imgur.com/IW50W.png}{style=max-width: 100\%; width: 444px;}

\caption{Polarization modes of a gravitational wave.\label{Figure: polarization of a gravitational wave}}
\end{figure}


\subsection{Detection of gravitational waves with Michelson interferometer}

A \href{https://en.wikipedia.org/wiki/Michelson_interferometer}{Michelson interferometer}
(like the one used to prove that there was no ether) is ideal to detect
changes in length in two orthogonal directions. It compares the time
it takes light to travel each of the two arms. 

For light we have 
\[
\DIFERENTIAL s^{2}=0\text{.}
\]
 From general relativity we also have 
\[
\DIFERENTIAL s^{2}=g_{\mu\nu}\DIFERENTIAL x^{\mu}\DIFERENTIAL x^{\nu}\text{.}
\]
 Let's consider the Michelson interferometer aligned with its arms
in the $\VERSOR x$ and $\VERSOR y$ direction. If we look what happens
to the light only in the $\VERSOR x$ direction then $\DIFERENTIAL x^{\mu}$
is $\ne0$ only for $\mu=1$ so 
\[
0=g_{11}\DIFERENTIAL x\DIFERENTIAL x\text{.}
\]
 Replacing \ref{Equation: metric with perturbation from a wave} this
is
\[
0=-c^{2}\DIFERENTIAL t^{2}+\PARENTHESES{1+h_{11}\PARENTHESES{\omega t-\VECTOR k\cdot\VECTOR x}}\DIFERENTIAL x^{2}\text{.}
\]
This can be seen as a modulation to the distance\footnote{Can it also be seen as a modulation to the speed of light? I mean:
$\frac{\DIFERENTIAL x}{\DIFERENTIAL t}=\frac{c}{\sqrt{1+h_{11}\PARENTHESES{\omega t-\VECTOR k\cdot\VECTOR x}}}$.}. We can integrate $\DIFERENTIAL t$ to see how much time it takes
light to travel some distance: 
\begin{align*}
\intop_{0}^{\tau_{\text{out}}}\DIFERENTIAL t & =\frac{1}{c}\intop_{0}^{L}\sqrt{1+h_{11}}\DIFERENTIAL x\\
 & \approx\frac{1}{c}\intop_{0}^{L}\PARENTHESES{1+\frac{1}{2}h_{11}\PARENTHESES{\omega t-\VECTOR k\cdot\VECTOR x}+\dots}\DIFERENTIAL x
\end{align*}
 where in the second line $\sqrt{1+x}\approx1+\frac{x}{2}+\dots$.
This is the time it takes light to travel from $x=0$ to the end of
the $\VERSOR x$ arm at $x=L$. There, there is a mirror so it bounces
back so we have the same expression repeated. Thus 
\begin{equation}
\tau_{0\to L\to0}=\frac{2L}{c}+\frac{1}{2c}\intop_{0}^{L}h_{11}\PARENTHESES{\omega t-\VECTOR k\cdot\VECTOR x}\DIFERENTIAL x\text{.}\label{Equation: tau 0 L 0 in x}
\end{equation}

For the interferometer we are interested in the difference in time
it takes light to travel each arm, i.e. 
\[
\Delta\tau=\tau_{0\to L\to0}^{\VERSOR x}-\tau_{0\to L\to0}^{\VERSOR y}\text{.}
\]
The expression in \ref{Equation: tau 0 L 0 in x} is for light traveling
in the $\VERSOR x$ direction. In the $\VERSOR y$ direction it is
the same but with $h_{22}$ instead of $h_{11}$. If we now consider
(to make it simple) a wave propagating in the $\VERSOR z$ direction
then we have 
\[
h_{11}=-h_{22}=h
\]
 so 
\[
\Delta\tau=\frac{1}{c}\intop_{0}^{L}h\PARENTHESES{\omega t-\VECTOR k\cdot\VECTOR x}\DIFERENTIAL x\text{.}
\]
We can also consider the limit 
\[
\omega\tau_{0\to L\to0}\ll1
\]
which is the same as considering the wavelength of the gravitational
wave much bigger than the length of the arms of the interferometer
or 
\[
kL\ll1\text{. }
\]
 In this limit we can regard $h$ as constant along the integration,
i.e. 
\[
h\PARENTHESES{\omega t-\VECTOR k\cdot\VECTOR x}\approx h\PARENTHESES{\omega t}
\]
 and so $\intop_{0}^{L}h\DIFERENTIAL x=hL$. So finally 
\begin{equation}
\Delta\tau=h\PARENTHESES t\frac{2L}{c}\text{.}\label{Equation: sa7td89gasdoasgyou}
\end{equation}
 $\Delta\tau$ is basically the signal that we will measure, so increasing
$L$ also increases the signal\footnote{``Increasing $L$ also increases the signal'' is true in the $L\ll\lambda$
limit. Anyway, for a $1\KILO{Hz}$ signal we have $\lambda_{1\KILO{Hz}}\approx300\KILO m$
so current detectors ($\sim5\KILO m$) are still well within that
limit.}. That is the reason for making very big interferometers. The limiting
factor for the size of the current detectors is the money~\cite{Reference: Saulson 2013 gravitational wave detection principles and practice}.

\subsubsection{The freely falling masses}

As stated before, a gravitational wave induces a relative movement
between freely falling masses. In the Michelson interferometer these
masses should be the beam splitter and the mirrors in the end of each
arm. If these elements are attached to the ground, they are the opposite
to ``freely falling masses''. If this were the case, when the gravitational
wave arrives it will try to change the distance between them but the
other forces would compensate this and they would not move. So the
mirrors and the beam splitter have to be mounted in some special way
such that they can be regarded as free masses. 

To accomplish this they are suspended like pendulums (sophisticated
pendulums). In the low angle regime the equation of motion for the
pendulum is $m\ddot{x}+kx=F_{\text{external}}$ which in the \uline{frequency
domain} can be written as 
\begin{equation}
\PARENTHESES{-\frac{m}{k}\omega^{2}+1}x=\frac{1}{k}F_{\text{external}}\text{.}\label{Equation: pendulum in frequency domain}
\end{equation}
 Note here that for $\omega\gg\sqrt{\frac{k}{m}}$ the dominant term
is $m\ddot{x}$ so it behaves as a free mass while for $\omega\ll\sqrt{\frac{k}{m}}$
the $kx$ term dominates and it keeps the mass fixed at a mean position. 

\subsubsection*{Why pendulums and not springs?}

The equation \ref{Equation: pendulum in frequency domain} describes
a mass attached to a spring, which is the small angle limit for a
pendulum. Why using pendulums instead of springs? The reason is to
reduce noise. In the case of the pendulum the $\VERSOR x$ component
of the restoring force is a projection of the gravitational force
exerted by the Earth, which is a conservative force. In the case of
a spring, it is the mean force resulting after adding all the microscopic
forces from the particles that make up the spring. Thus, it is to
be expected that the spring is more noisy. More details in~\cite{Reference: Saulson 2013 gravitational wave detection principles and practice}.

\subsection{The importance of multiple observations}

The ``\href{https://en.wikipedia.org/wiki/Radiation_pattern}{radiation pattern}''
for a gravitational wave detector looks like the one shown in \ref{Figure: gravitational wave detector radiation pattern}.
It is remarkably isotropic~\cite{Reference: Saulson 2013 gravitational wave detection principles and practice}.
This makes it impossible to tell where the wave is coming from\footnote{The opposite to this is an optical telescope that has a radiation
pattern that is almost a Dirac delta pointing in the direction of
observation. }. Multiple observations are required to tell where an event is coming
from using the arrival time of the wave to each observatory and the
distance between them:
\begin{itemize}
\item Observation at one single point: You cannot tell anything about where
the wave is coming from.
\item Observation at two points: It is possible to determine a circumference
in the sky.
\item Observation at three points: Determines the intersection of two circles
so the event is coming from one of two points.
\item Observation at four points: Determines the position unambiguously. 
\end{itemize}
\begin{figure}[H]
\htmltag{tag_name=image}{src=media/1.svg}{style=max-width: 100\%;}

\caption{Response of a gravitational wave interferometer as a function of direction,
averaged over polarizations. Image copy-pasted from \cite{Reference: Saulson 2013 gravitational wave detection principles and practice}.
\label{Figure: gravitational wave detector radiation pattern}}
\end{figure}


\subsection{Noise sources}

The signal produced by gravitational waves in gravitational detectors
is very small, we are talking of~\cite{Reference: Saulson 2013 gravitational wave detection principles and practice}
\begin{equation}
\frac{\Delta L}{L}\sim10^{-22}\text{.}\label{Equation: DeltaL sobre L que queremos medir}
\end{equation}
 Thus, it is crucial to understand and mitigate the noise sources. 

\subsubsection{Shot noise}

A Michelson interferometer can be seen as a transducer that converts
differences in length to differences in light power at its output.
If the wavelength of the light is kept constant (and so the energy
of the photons) then measuring the power is equivalent to counting
the number of photons per unit time received. These photons follow
a Poisson distribution with expectation value $\frac{P_{\text{out}}}{h\nu}$.
This introduces a fundamental source of noise with origin in quantum
fluctuations. 

When $\frac{P_{\text{out}}}{h\nu}\ggg1$ the Poisson distribution
can be approximated by a Gaussian with $\mu=\sigma^{2}=\frac{P_{\text{out}}}{h\nu}$.
In this limit we have 
\[
\frac{\sigma}{\mu}=\sqrt{\frac{h\nu}{P_{\text{out}}}}
\]
 so making $P_{\text{out}}$ as large as possible reduces the shot
noise. There are a number of ways of accomplishing this, by increasing
the laser power, using a \emph{power recycling technique}, etc. 

\subsubsection{Thermal noise}

As every system in the universe, here we also have thermal noise.
It is indeed one of the strongest noise sources~\cite{Reference: Saulson 2013 gravitational wave detection principles and practice}.
Thermal noise in a gravitational wave detector is a generalization
of the \emph{Brownian motion}. Each of the nearly free mirrors can
be regarded as a very big Brownian particle. Air molecules hitting
on them would induce small Brownian motion, that is why the mirrors
are suspended in vacuum. 

There are still other sources that can induce Brownian motion. One
of them is related to the materials used to hold the mirrors. This
explains why mirrors are held with pendulum-like systems: the restoring
force is the horizontal projection of gravity which, as a conservative
force, lacks of thermal noise~\cite{Reference: Saulson 2013 gravitational wave detection principles and practice}.
This makes the thermal noise coming from the holding mechanism negligible. 

The biggest contribution to thermal noise are the internal degrees
of freedom of the mirrors. Basically there are vibrations that introduce
a noise spectrum starting at $10\KILO{Hz}$. To reduce this source
of noise there are basically two options: 1) use ``the best possible
material'' for the mirrors and 2) lower the temperature. Current
interferometers use mirrors made of \emph{fused silica} at room temperature.
Future interferometers are planning to use cryogenic systems instead. 

Another source of thermal noise is the coating of the mirrors. 

\subsubsection{Seismic noise}

It is not hard to understand what this means and/or where it comes
from. When seeking for such small displacements, every vibration that
the Earth makes is important. Apparently, every part of the Earth's
surface is continuously shaking with amplitudes of $\sim1\MICRO m$
on time scales of seconds. If we are trying to measure magnitudes
as stated in \ref{Equation: DeltaL sobre L que queremos medir} with
$L\sim1\KILO m$, this $1\MICRO m$ shakes are almost infinite. There
are two possibilities to fight against this noise: 1) somehow isolate
our system and 2) go deep underground where the seismic noise is smaller. 

Current detectors are all at ground level and use sophisticated pendulum
systems to isolate the seismic noise. The best seismic noise attenuator
to date (2013) is the one used in the Virgo detector, shown in \ref{Figure: Virgo super attenuator}.

\begin{figure}[H]
\htmltag{tag_name=image}{src=media/2.svg}{style=max-width: 100\%;}

\caption{Taken from~\cite{Reference: Saulson 2013 gravitational wave detection principles and practice}.
\label{Figure: Virgo super attenuator}}
\end{figure}


\subsubsection{Other noises of source}

Of course there are other sources of noise, in the electronics, in
the control systems, in the intensity and wavelength of the laser,
etc. 

\section*{30.09.2021}

Gravitational waves arise in the linearized theory. The speed of gravity
is the same as the speed of light. General relativity is highly non
linear
\[
R\COVARIANTINDEX{\mu\nu}=-\frac{1}{2}g\COVARIANTINDEX{\nu\nu}R=\frac{8\pi G}{c^{4}}T\COVARIANTINDEX{\mu\nu}
\]
 Now we consider a weak field approximation, basically very far away
from the sources. The waves are just perturbations to a time-independent
background curvature. For this to hold the curvature radius has to
be much greater than the wavelength of the gravitational waves\footnote{Does this mean that even on Earth (i.e. very far away from sources
and in a relatively weak field) there is a regime for non linear waves?}: 
\[
\text{Time independent curvature radius}\gg\text{wavelength}
\]
 In this conditions the metric tensor can be written as 
\[
g\COVARIANTINDEX{\mu\nu}=\eta\COVARIANTINDEX{\mu\nu}+h\COVARIANTINDEX{\mu\nu}
\]
 where $\eta$ is the Minkowski and $h$ is a small perturbation with
$\MODULE{h\COVARIANTINDEX{\mu\nu}}\ll1$. Because our space is esentialy
flat (linearized theory) then it holds 
\[
h\CONTRAVARIANTINDEX{\mu\nu}=\eta\CONTRAVARIANTINDEX{\mu\alpha}\eta\CONTRAVARIANTINDEX{\nu\beta}h\COVARIANTINDEX{\alpha\beta}
\]
 and so the inverse metric is 
\[
h\CONTRAVARIANTINDEX{\mu\nu}=\eta\CONTRAVARIANTINDEX{\mu\nu}-h\CONTRAVARIANTINDEX{\mu\nu}\text{.}
\]
 Now I see ``trace reversed'' and it says 
\[
\bar{h}\COVARIANTINDEX{\mu\nu}=h\COVARIANTINDEX{\mu\nu}-\frac{1}{2}\eta\COVARIANTINDEX{\mu\nu}h
\]
 where $h=h\COVARIANTINDEX{\mu\nu}h\CONTRAVARIANTINDEX{\mu\nu}$.
I don't know what this is. 

There is a gauge freedom given by the infinitesimal transformation
\[
x\CONTRAVARIANTINDEX{\mu}\to x\PRIME\CONTRAVARIANTINDEX{\mu}=x\CONTRAVARIANTINDEX{\mu}+\xi\CONTRAVARIANTINDEX{\mu}\PARENTHESES x
\]
 and this holds if $\MODULE{\partial\COVARIANTINDEX{\mu}\xi\COVARIANTINDEX{\nu}}\ll\MODULE{h\COVARIANTINDEX{\mu\nu}}$.
This transformation modifies $h$ to 
\[
h\COVARIANTINDEX{\mu\nu}\PARENTHESES x\to h\PRIME\COVARIANTINDEX{\mu\nu}\PARENTHESES x=h\COVARIANTINDEX{\mu\nu}\PARENTHESES x-\PARENTHESES{\partial\COVARIANTINDEX{\mu}\xi\COVARIANTINDEX{\nu}-\partial\COVARIANTINDEX{\nu}\xi\COVARIANTINDEX{\mu}}
\]
Cannot read something in blue, next: $\MODULE{h\PRIME\COVARIANTINDEX{\mu\nu}}\ll1$.
Now we choose the Lorentz (Hilbert) gauge which means 
\[
h\COVARIANTINDEX{\mu\nu,\nu}=0
\]
 and as a result we have (the Einstein equation is this?)
\[
\PARENTHESES{-\frac{\partial^{2}}{\partial^{2}t^{2}}+\nabla^{2}}\bar{h}\CONTRAVARIANTINDEX{\mu\nu}=-16\pi T\CONTRAVARIANTINDEX{\mu\nu}
\]
 where now we are using 
\[
G=c=1\text{.}
\]
 In vacuum 
\[
\square\bar{h}\CONTRAVARIANTINDEX{\mu\nu}=0
\]
 so this means that we will have plane wave solutions of the form
\[
\bar{h}\CONTRAVARIANTINDEX{\mu\nu}=\mathscr{A}e\CONTRAVARIANTINDEX{\mu\nu}e^{ik\COVARIANTINDEX{\alpha}k\CONTRAVARIANTINDEX{\alpha}}
\]
 where $\mathscr{A}$ is the amplitude, $e\CONTRAVARIANTINDEX{\mu\nu}$
is the polarization tensor and $e^{i\dots}$ is an exponential. $k\CONTRAVARIANTINDEX{\alpha}$
is the wave vector which is lightlike, i.e. 
\[
k\COVARIANTINDEX{\mu}k\CONTRAVARIANTINDEX{\mu}=0\GRAY{\leftarrow\text{Lightlike}}
\]
 (because the wave travels at speed $c$). The polarization and wave
vector are orthogonal, this means that 
\[
e\CONTRAVARIANTINDEX{\mu\nu}k\COVARIANTINDEX{\nu}=0\GRAY{\leftarrow\text{Transverse waves}}
\]
 so this means that the waves are transverse. The degrees of freedom
are 6 in total, I did not follow the explanation. The calculation
is 
\[
16\text{ components and gauge conditions}\rightarrow10-4=6\text{.}\GRAY{\leftarrow\text{Degrees of freedom}}
\]
 

The waves are \uline{transverse and traceless}. Transverse wave
means 
\[
e\CONTRAVARIANTINDEX{ij}k\COVARIANTINDEX j=0\GRAY{\leftarrow\text{Transverse waves}}
\]
 and this comes from $e\CONTRAVARIANTINDEX{0p}=0$ (?). Traceless
wave means that 
\[
e\CONTRAVARIANTINDEX i\COVARIANTINDEX i=0\GRAY{\leftarrow\text{Traceless waves}}
\]
 and I am not sure where this comes from. We can use a transverse
and traceless (TT) gauge for the metric and this will give us that
there are two independent polarizations and trace-reversed metric
\[
\bar{h}\COVARIANTINDEX{\mu\nu}=h\COVARIANTINDEX{\mu\nu}\text{.}
\]

Let's now consider a wave propagating in the $\VERSOR z$ direction.
This means that 
\[
\text{Wave in the }z\text{ direction}\rightarrow\LBRACE{\begin{aligned} & k\COVARIANTINDEX z=\omega\\
 & k\CONTRAVARIANTINDEX 0=\omega\\
 & k\COVARIANTINDEX x=k\COVARIANTINDEX y=0\\
 & e\CONTRAVARIANTINDEX{0\mu}=e\CONTRAVARIANTINDEX{z\mu}=0\\
 & e\CONTRAVARIANTINDEX{xx}=-e\CONTRAVARIANTINDEX{yy}
\end{aligned}
}
\]
 The independent components are 
\[
\GRAY{\text{Two independent components}\rightarrow}\LBRACE{\begin{aligned} & e\CONTRAVARIANTINDEX{xx}\DEF e^{\oplus}\equiv\text{plus polarization}\\
 & e\CONTRAVARIANTINDEX{xy}\DEF e^{\otimes}\equiv\text{cross polarization}
\end{aligned}
}
\]

If now we assume $e\CONTRAVARIANTINDEX{xy}=0$ then we only have $\oplus$
polarization, i.e. a ``pure $\oplus$ metric''. In this case the
line element has the form 
\[
\DIFERENTIAL s^{2}=-\DIFERENTIAL t^{2}+\PARENTHESES{1+h_{\oplus}}\DIFERENTIAL x^{2}+\PARENTHESES{1-h_{\oplus}}\DIFERENTIAL y^{2}+\DIFERENTIAL z^{2}
\]
 with $h_{\oplus}=\mathscr{A}e\CONTRAVARIANTINDEX{xx}e^{i\omega\PARENTHESES{t-z}}$. 

If we have a wave with $e\CONTRAVARIANTINDEX{xx}=0$ then we have
``pure $\otimes$ polarization'' and we have the same but with a
45 degrees rotation. 

The effect of each of these polarizations on a ring of free particles
is shown in \ref{Figure: polarization of a gravitational wave}. 

\begin{figure}[H]
\begin{centering}
\htmltag{tag_name=image}{src=https://upload.wikimedia.org/wikipedia/commons/thumb/0/0d/Quadrupol_Wave.gif/330px-Quadrupol_Wave.gif}{style=max-width: 100\%;}
\par\end{centering}
\caption{Effect of a gravitational wave propagating in the direction orthogonal
to the screen.}
\end{figure}

We can also define a circular polarization basis as 
\[
\GRAY{\text{Circular polarization}\rightarrow}\LBRACE{\begin{aligned} & e_{R}=\frac{1}{\sqrt{2}}\PARENTHESES{e_{\oplus}+ie_{\otimes}}\\
 & e_{L}=\frac{1}{\sqrt{2}}\PARENTHESES{e_{\oplus}-ie_{\otimes}}
\end{aligned}
}
\]
 

Let's look now a the geodesic equation: 
\[
\frac{\DIFERENTIAL^{2}x\CONTRAVARIANTINDEX{\alpha}}{\DIFERENTIAL\tau^{2}}+\Gamma_{\mu\nu}^{\alpha}\frac{\DIFERENTIAL x\CONTRAVARIANTINDEX{\mu}}{\DIFERENTIAL\tau}\frac{\DIFERENTIAL x\CONTRAVARIANTINDEX{\nu}}{\DIFERENTIAL\tau}=0\GRAY{\leftarrow\text{Geodesic equation}}
\]
 If we use the TT gauge (see before) this is 
\[
\frac{\DIFERENTIAL^{2}x\CONTRAVARIANTINDEX i}{\DIFERENTIAL\tau^{2}}=-\Gamma_{00}^{i}=-\frac{1}{2}\PARENTHESES{2h\COVARIANTINDEX{i0,0}-h\COVARIANTINDEX{00,i}}=0
\]
 so this means that the particle is not accelerated. This means that 
\begin{itemize}
\item the TT gauge defines a coordinate system that is co-moving with freely
falling test particles\footnote{This means following the motion induced by the wave?}. 
\item TT time is the proper time on a clock of a freely falling test particle
($h\COVARIANTINDEX{0\mu}=0$). 
\end{itemize}
%
Now we look at the equation of geodesic deviation, which (I believe)
means how the separation of neighboring particles changes, i.e. it
somehow means the tidal forces. This equation is 
\[
\frac{\DIFERENTIAL^{2}\xi}{\DIFERENTIAL\tau^{2}}=R\CONTRAVARIANTINDEX i\COVARIANTINDEX{0j0}\xi\CONTRAVARIANTINDEX j
\]
 where $R$ is the oscillating curvature tensor from a passing wave
\[
R\CONTRAVARIANTINDEX i\COVARIANTINDEX{0j0}=\frac{1}{2}h^{\PARENTHESES{TF?}}\CONTRAVARIANTINDEX i\COVARIANTINDEX{j,00}
\]
 with initially $\xi\CONTRAVARIANTINDEX j\PARENTHESES 0=$ something
with $k$ that my eyes cannot see. Somehow this will induce a tidal
force between the two neighboring particles and so the total force
will be 
\[
F_{\text{total}}=F_{\text{gravitational wave}}+F_{\text{others}}\text{.}
\]
 

\subsection*{Detection of gravitational waves}

See slides. We can rewrite \ref{Equation: sa7td89gasdoasgyou} as
\[
\frac{h}{2}=\frac{c\Delta t}{L}
\]
 and here we identify 
\[
\frac{d\Delta t}{L}=\frac{\Delta L}{L}
\]
 and from theory we have 
\[
\frac{h}{2}\sim10^{-21}
\]
 and so we see that this problem is really complicated, and we can
estimate how to build our detectors. 

\section*{07.10.2021}

\section{Gravitational wave sources}

In principle everything is a source, but we will consider only sources
that can produce waves detectable by detectors. 

\subsection{Binary systems}

This is usually a last problem in a GR course. There are two point
masses of mass $m$ both which are orbiting. As a result of gravitational
radiation the energy is lost and the distance between the particles
gets smaller, this increases $\omega$, this increases the radiation
(i.e. the amplitude), and so on. In reality the point masses are not
point and at some point they collide and become one single mass. The
results are summarized in the slides. 

\subsubsection{Binary pulsars}

With pulsars we can see at the period shift. See \href{https://www.google.com/search?q=pulsar+period+shift&client=ubuntu&hs=8dn&channel=fs&sxsrf=AOaemvLE6YV8Jd4B1_ID5JYaMl6pcmAszw:1633597052054&source=lnms&tbm=isch&sa=X&ved=2ahUKEwjMiPmf97fzAhUSP-wKHeD6AHAQ_AUoAXoECAEQAw&biw=1920&bih=969&dpr=1\#imgrc=r_Rt-hFumJNK8M}{here}.

\subsubsection{Black hole binaries}

Emission by black holes is actually easier to model than that of neutron
stars because there is no matter. The models for sources with black
holes are valid for any mass, it does not matter. With neutron stars
this is not the case because if you have a model for neutron stars
of mass $m$ it may not hold for mass $M$ because many things inside
the neutron star change. 

The higher frequency a black hole can emit is given by the ``quasi
normal modes'' obtained by perturbating the surface of a black hole. 

The peak frequency of a BNS system is lower than that of black holes.
The reason is that the neutron stars are much bigger in size, so they
``touch'' before than the two black holes. They also have much more
complicated quasi normal modes which have lower frequency. 

There are different flavors of black holes:
\begin{itemize}
\item Total mass. Depending on the total mass of the binary we can classify
them like this:
\begin{itemize}
\item Stellar mass BH. This is when $3M_{\sun}<M_{BH}<100M_{\sun}$. These
are black holes formed from the decay of a star.
\item Intermediate mass BH. When $100M_{\sun}<M_{BH}<10^{5}M_{\sun}$. These
black holes do not form from supernova explosions. One mechanism for
these black holes is merging of multiple black holes.
\item Super massive BH. Those usually found in the center of the galaxies. 
\item Near solar mass BH. $M_{BH}<3M_{\sun}$. These have not been observed
yet. It would be a break through for dark matter. There is a minimum
mass for black holes given by Hawking radiation, it it is smaller
they would have already evaporate.
\end{itemize}
\end{itemize}
The astrophysical sources of gravitational waves by binary black holes
are intermediate mass black holes. This usually happens in the center
of galaxies: The density of objects is bigger and also older so stars
have already converted into black holes and black holes have already
merged. 

The higher the mass, the lower the peak frequency. Even if this may
appear counter intuitive, the reason is that if there is more mass
then the surface of the black hole is bigger and so the quasi normal
modes (oscillations of its surface) have longer wavelength so lower
frequency. 

\subsection{Supernovae }

The explosion of a star is not spherically symmetric (because the
star rotates, and other effects) and because it is a very energetic
process, it can emit gravitational waves detectable. The issue is
that there is not a model for this type of wave basically because
a supernova is a mess. 

The detection of gravitational waves from supernovae (or neutrinos)
is very important to understand what kind of object is left after
such an event. The reason is that gravitational waves (and neutrinos)
travel directly towards us without interactions, so it is a clean
signal of what has happened. Light, which is what we use now to observe
these events, scatters a lot with the matter of the dying star so
it is very distorted. 

\subsection{Rotating neutron stars}

Rotating neutron stars emit very weak but continuous signals. If you
take data for many time (say 10 years) you can integrate the power
of this signal over all your period. 

\begin{figure}[H]
\begin{centering}
\htmltag{tag_name=image}{src=https://astrobites.org/wp-content/uploads/2018/01/nonaxisymmetric_neutron_star_sketch_lowres1.gif}{style=max-width: 100\%;}
\par\end{centering}
\caption{Emission of gravitational waves by a rotating neutron star. }
\end{figure}


\subsection{Stochastic background}

This signal is supposed to be constantly present, though it has not
a fixed frequency (like rotating neutron stars) but rather a continuous
spectrum. This is like the CMB background. 

We can classify this background into two categories:
\begin{enumerate}
\item Population of point sources. This means that there are many events
that happened in the history of the universe, and we see all of them
superimposed. Analogy: the noise that the drops of water produce when
it rains; you listen a continuous noise and not the individual events.
This is expected to be between $10\UNIT{Hz}$ and $4\KILO{Hz}$.
\item Cosmological. The anisotropies observed in the CMB are related with
differences in the density of energy. This must have produced somehow
a gravitational wave background similar to the CMB. Because this background
is very red shifted, the peak frequency is very low to be observed
nowadays. There are other models which propose a very high frequency
despite the red shift. We really don't know yet. 
\end{enumerate}

\section*{14.10.2021}

\section{Source modeling}

Everything from today was taken from \cite{Reference: Jolien Gravitational Wave Physics and Astronomy BOOK}
chapter 3.

We are going to see the modeling of sources in a Newtonian limit.
The non-Newtonian limit is very cumbersome.

We have seen that if there is a rigid body with some extension in
space, then we can measure a gravitational wave by measuring the heat
that it will produce due to friction in this body. Gravitational waves
have no energy locally (?). 

For a gravitational wave we consider a metric of the form 
\[
g\COVARIANTINDEX{\mu\nu}=\eta\COVARIANTINDEX{\mu\nu}+h\COVARIANTINDEX{\mu\nu}
\]

where 
\[
g\COVARIANTINDEX{ij}=\eta\COVARIANTINDEX{ij}+\nu h^{\PARENTHESES 1}\COVARIANTINDEX{ij}+\nu^{2}h^{\PARENTHESES 2}\COVARIANTINDEX{ij}+\dots
\]
 This allows us to compute the Einstein tensor as a series 
\begin{align*}
G\COVARIANTINDEX{\mu\nu} & =R\COVARIANTINDEX{\mu\nu}-\frac{1}{2}g\COVARIANTINDEX{\mu\nu}R\\
 & =\GUNDERBRACE{G\COVARIANTINDEX{ij}}{=0}+G^{\PARENTHESES 1}\COVARIANTINDEX{ij}+G^{\PARENTHESES 2}\COVARIANTINDEX{ij}+\dots
\end{align*}
 EFE (Einstein Field Equation?) in vacuum 
\[
G\COVARIANTINDEX{\mu\nu}=0\Rightarrow0\approx\nu G^{\PARENTHESES 1}\COVARIANTINDEX{ij}\PARENTHESES{h^{\PARENTHESES 1}\COVARIANTINDEX{ij}}+\GUNDERBRACE{\nu^{2}\SQBRACKETS{G^{\PARENTHESES 2}\COVARIANTINDEX{ij}\PARENTHESES{h^{\PARENTHESES 1}\COVARIANTINDEX{ij}}+G^{\PARENTHESES 1}\COVARIANTINDEX{ij}\PARENTHESES{h^{\PARENTHESES 2}\COVARIANTINDEX{ij}}}}{=0}
\]
 $\ORDER{h^{2}}$: effective stress-energy tensor $\rightarrow$ this
is the source term for $h^{\PARENTHESES 2}\COVARIANTINDEX{ij}$ contributions.
I think that now we replace this back into the Einstein equation to
find the stress-energy carried by the gravitational wave 
\[
G^{\PARENTHESES 1}\COVARIANTINDEX{ij}\PARENTHESES{h^{\PARENTHESES 2}\COVARIANTINDEX{ij}}=\frac{8\pi G}{c^{4}}T^{\text{GW}}\COVARIANTINDEX{ij}\GRAY{\leftarrow\text{Stress-energy of a GW}}
\]
 We can also replace the $h^{\PARENTHESES 2}$ by $G^{\PARENTHESES 2}\COVARIANTINDEX{ij}\PARENTHESES{h^{\PARENTHESES 1}\COVARIANTINDEX{ij}}$
from the previous equation (I'm not following) so 
\[
T^{\text{GW}}\COVARIANTINDEX{ij}=\frac{c^{4}}{8\pi G}G^{\PARENTHESES 2}\PARENTHESES{h^{\PARENTHESES 1}\COVARIANTINDEX{ij}}
\]
 and this allows us to create a stress-energy for a gravitational
wave with first order $h\COVARIANTINDEX{ij}$. 

\subsection*{Gauge invariance}

We take the average over a region of space-time that contains several
gravitational wave wavelengths. We take the following average: 
\[
\frac{c^{4}}{8\pi G}\ANGLEBRACKETS{R^{\PARENTHESES 2}\COVARIANTINDEX{ij}\PARENTHESES{h^{\PARENTHESES 1}\COVARIANTINDEX{ij}}-\frac{h\COVARIANTINDEX{ij}}{2}R^{\PARENTHESES 2}}
\]
 $R^{\PARENTHESES 2}\COVARIANTINDEX{ij}\PARENTHESES{h^{\PARENTHESES 1}\COVARIANTINDEX{ij}}$
has quadratic terms in $\Gamma$. Now I have no idea what we use,
but we obtain 
\[
T^{\text{GW}}\COVARIANTINDEX{ij}=\frac{c^{4}}{32\pi G}\ANGLEBRACKETS{\frac{\partial h_{TT}\CONTRAVARIANTINDEX{\alpha\beta}}{\partial x\COVARIANTINDEX i}\frac{\partial h_{TT}\COVARIANTINDEX{\alpha\beta}}{\partial x\COVARIANTINDEX j}}
\]
 where 
\begin{align*}
h_{TT}\COVARIANTINDEX{\alpha\beta} & =h_{TT}\COVARIANTINDEX{\alpha\beta}\PARENTHESES{t-\frac{z}{c}}\\
 & =h_{+}e_{+}\COVARIANTINDEX{\alpha\beta}+h_{\times}e_{\times}\COVARIANTINDEX{\alpha\beta}
\end{align*}
 is a wave propagating in the $z$ direction and $h_{TT}$ I think
it means $h_{\text{Transverse-Traceless}}$, but not sure. Thus 
\[
T^{\text{GW}}\COVARIANTINDEX{00}=\frac{c^{4}}{16\pi G}\ANGLEBRACKETS{\dot{h}_{+}^{2}+\dot{h}_{\times}^{2}}
\]
 
\[
-cT\COVARIANTINDEX{03}=-cT\COVARIANTINDEX{30}=c^{2}T\COVARIANTINDEX{33}=T^{\text{GW}}\COVARIANTINDEX{00}
\]
 and all other components are zero. (We used that $e_{+}\CONTRAVARIANTINDEX{ij}e_{\times}\COVARIANTINDEX{ij}=e_{\times}\CONTRAVARIANTINDEX{ij}e_{+}\COVARIANTINDEX{ij}=2$.)
This is the ``gravitational wave flux''. 

El aula tiene 3 pizarrones de los cuales s�lo se pueden ver 2 en simult�neo.
Ahora escondi� el pizarr�n 2 y arranc� en el 3, as� que no puedo ver
nada de lo que acaba de escribir. We are interested in solving the
Einstein field equations for this $T$. Me limito a una tarea de copista
de ahora en adelante, sin interpretaci�n. 
\[
h_{ij,j}=0\rightarrow\boxed{\tau\CONTRAVARIANTINDEX{ij}\COVARIANTINDEX{ij}=0}\text{ conservation law for coordinate system}
\]
 
\[
\square\bar{h}\COVARIANTINDEX{ij}\PARENTHESES{t,\VECTOR x}=\frac{4G}{c^{4}}\intop\frac{\tau_{ik}\PARENTHESES{t-\frac{\MODULE{\VECTOR x-\VECTOR x\PRIME}}{c},\VECTOR x\PRIME}}{\MODULE{\VECTOR x-\VECTOR x\PRIME}}\DIFERENTIAL^{3}x\PRIME
\]
This is the integral we have to solve. We will solve this in two regimes:
\begin{itemize}
\item Far-zone: Means that $R\ll\lambda\ll r$ where $R$ is the source
size and $r$ the distance we are looking from.
\item Near-zone: Means $R\ll r\ll\lambda$.
\end{itemize}

\subsection{Far zone}

We are using $R\lll r$ so the distance to the source is always the
same: 
\[
\MODULE{\VECTOR x-\VECTOR x\PRIME}\approx r
\]
 We consider also slow motion: 
\[
t-\frac{\MODULE{\VECTOR x-\VECTOR x\PRIME}}{c}\sim t-\frac{r}{c}
\]
 which in some way is a consequence from the previous assumption.
In this conditions we have 
\[
\bar{h}\COVARIANTINDEX{ij}=\frac{4G}{c^{4}r}\intop\tau\COVARIANTINDEX{ik}\PARENTHESES{t-\frac{r}{c},\VECTOR x\PRIME}\DIFERENTIAL^{3}x\PRIME
\]
 Now we approximate 
\[
\tau\CONTRAVARIANTINDEX{\alpha\beta}\approx\frac{1}{2}\frac{\partial^{2}}{\partial t^{2}}\PARENTHESES{x\CONTRAVARIANTINDEX{\alpha}x\CONTRAVARIANTINDEX{\beta}\tau\CONTRAVARIANTINDEX{00}}
\]
 find 
\[
\bar{h}\COVARIANTINDEX{\alpha\beta}=\frac{2G}{c^{4}r}\frac{\partial^{2}}{\partial t^{2}}\GUNDERBRACE{\intop x\PRIME\CONTRAVARIANTINDEX{\alpha}x\PRIME\CONTRAVARIANTINDEX{\beta}\tau\CONTRAVARIANTINDEX{00}\PARENTHESES{t-\frac{r}{c},\VECTOR x}\DIFERENTIAL^{3}x\PRIME}{\text{quadrupole tensor of GW}}
\]
 
\[
\bar{h}\CONTRAVARIANTINDEX{\alpha\beta}\sim\frac{2G}{c^{4}r}\ddot{I}\CONTRAVARIANTINDEX{\alpha\beta}\PARENTHESES{t-\frac{r}{c}}
\]
 Now we project into TT gauge $P\COVARIANTINDEX{\alpha\beta}=\delta\COVARIANTINDEX{\alpha\beta}-\hat{\eta}\COVARIANTINDEX{\alpha}\hat{\eta}\COVARIANTINDEX{\beta}$
and finally the solution in the near zone is 
\[
\boxed{\bar{h}_{TT}\COVARIANTINDEX{\alpha\beta}\sim\frac{2G}{c^{4}r}\ddot{I}_{TT}\COVARIANTINDEX{\alpha\beta}}
\]
 

\subsection{Near zone}

In the near zone the gravitational potential to lowest order (Newtonian
limit) is 
\begin{align*}
\phi & =-\frac{c^{4}}{2}h\CONTRAVARIANTINDEX{00}\\
 & =-\frac{c^{4}}{4}\PARENTHESES{\bar{h}\CONTRAVARIANTINDEX{00}+\delta\COVARIANTINDEX{\alpha\beta}\frac{\bar{h}\CONTRAVARIANTINDEX{\alpha\beta}}{c^{2}}}
\end{align*}
Here there was a long discussion, and we ended up with this: 
\[
\phi=-G\intop\frac{\tau\CONTRAVARIANTINDEX{00}\PARENTHESES{t,\VECTOR x}+\frac{1}{c^{2}}\delta\COVARIANTINDEX{\alpha\beta}\tau\CONTRAVARIANTINDEX{\alpha\beta}\PARENTHESES{t,\VECTOR x}}{\MODULE{\VECTOR x-\VECTOR x\PRIME}}\DIFERENTIAL^{3}x\PRIME
\]
 Now we expand the denominator in powers of $\frac{1}{r}$ because
we are not in the far zone. (I think that $\alpha$ and $\beta$ run
only on the spacial components.)

Better to see this from the book~\cite{Reference: Jolien Gravitational Wave Physics and Astronomy BOOK}. 

\subsection{Gravitational radiation luminosity}

See \cite{Reference: Jolien Gravitational Wave Physics and Astronomy BOOK}
� 3.3.2.

The gravitational flux is the amount of energy carried by a gravitational
wave that crosses a spherical surface element surrounding the source
per unit time. It is 
\begin{align*}
\frac{\DIFERENTIAL E}{\DIFERENTIAL t\DIFERENTIAL A} & =T^{\text{GW}}\COVARIANTINDEX{03}\\
 & \vdots
\end{align*}
As a result we get 
\[
L_{\text{GW}}=-\frac{\DIFERENTIAL E}{\DIFERENTIAL t}
\]


\subsection{Emission by an orbiting binary system}

\cite{Reference: Jolien Gravitational Wave Physics and Astronomy BOOK}
� 3.5. We want to compute the quadrupolar moment 
\[
I\CONTRAVARIANTINDEX{ij}=\intop\DIFERENTIAL^{3}x\PARENTHESES{x\CONTRAVARIANTINDEX ix\CONTRAVARIANTINDEX j-\frac{r^{2}}{3}\delta\CONTRAVARIANTINDEX{ij}}\rho\PARENTHESES{t,\VECTOR x}
\]
 Now we go to the reduced mass problem 
\[
\LBRACE{\begin{aligned} & a=r_{1}+r_{2}\\
 & M=m_{1}+m_{2}\\
 & \mu=\frac{m_{1}m_{2}}{M}
\end{aligned}
}
\]
 Using this we get the expressions \cite{Reference: Jolien Gravitational Wave Physics and Astronomy BOOK}
(3.165). This is 
\[
\LBRACE{\begin{aligned} & I_{11}=\frac{\mu a^{2}}{2}\PARENTHESES{1+\cos\PARENTHESES{2\varphi}}\\
 & I\COVARIANTINDEX{22}=\frac{\mu a^{2}}{2}\PARENTHESES{1-\cos\PARENTHESES{2\varphi}}\\
 & I\COVARIANTINDEX{12}=\frac{\mu a^{2}}{2}\sin\PARENTHESES{2\varphi}
\end{aligned}
}
\]
 Now we want to calculate the perturbation to the metric, this is
\[
h^{TT}\COVARIANTINDEX{ij}\sim\frac{2G}{c^{4}r}\ddot{I}^{TT}\COVARIANTINDEX{ij}\PARENTHESES{t-\frac{r}{c}}
\]
 Replacing the previous components we obtain 
\[
h^{TT}\COVARIANTINDEX{ij}=-\frac{4G\mu a^{2}\omega^{2}}{c^{4}r}\SQBRACKETS{\begin{matrix}\cos2\varphi & \sin2\varphi & 0\\
\sin2\varphi & -\cos2\varphi & 0\\
0 & 0 & 0
\end{matrix}}
\]
 We have used in the beginning that the observer is in the $\VERSOR z$
direction, thus we already have the $h$ in the TT gauge. If we use
the polarization tensors 
\[
\LBRACE{\begin{aligned} & \VECTOR e_{+}=\SQBRACKETS{\begin{matrix}1 & 0 & 0\\
0 & -1 & 0\\
0 & 0 & 0
\end{matrix}}\\
 & \VECTOR e_{\times}=\SQBRACKETS{\begin{matrix}0 & 1 & 0\\
1 & 0 & 0\\
0 & 0 & 0
\end{matrix}}
\end{aligned}
}
\]
then each component has amplitude 
\[
\LBRACE{\begin{aligned} & h_{+}=-\frac{4G\mu a^{2}\omega^{2}}{c^{4}r}\cos2\varphi\\
 & h_{\times}=-\frac{4G\mu a^{2}\omega^{2}}{c^{4}r}\sin2\varphi
\end{aligned}
}
\]
 

Here we note that the frequency of the gravitational wave is twice
the frequency of the orbital motion: 
\[
f_{\text{gravitational wave}}=2f_{\text{orbital motion}}
\]

If we use 
\[
v\DEF a\omega
\]
 which is basically the velocity of the reduced mass orbiting body,
or similar, then we can rewrite 
\[
\LBRACE{\begin{aligned} & h_{+}=-\frac{4G\mu}{c^{2}r}\PARENTHESES{\frac{v}{c}}^{2}\cos2\varphi\\
 & h_{\times}=-\frac{4G\mu}{c^{2}r}\PARENTHESES{\frac{v}{c}}^{2}\sin2\varphi
\end{aligned}
}
\]
 

If the observer is not exactly at $\VERSOR z$ but at an angle $\iota$
(iota) the amplitudes are modified according to \cite{Reference: Jolien Gravitational Wave Physics and Astronomy BOOK}
eq. (3.172). 

Now we want to see how this system evolves in time. Basically the
gravitational radiation will carry energy, this will make the orbits
to reduce, frequency increase and so on. To study this we need the
gravitational wave luminosity, and for this we compute the third time
derivative of the quadrupole tensor. The result is 
\begin{align*}
L_{\text{gravitational wave}} & =\frac{1}{5}\frac{G}{c^{5}}\ANGLEBRACKETS{\PARENTHESES{\ddddot{I}\COVARIANTINDEX{11}}^{2}+\PARENTHESES{\dddot{I}\COVARIANTINDEX{22}}^{2}+2\PARENTHESES{\dddot{I}\COVARIANTINDEX{12}}^{2}}\\
 & =\frac{32}{5}\frac{c^{5}}{G}\eta^{2}\PARENTHESES{\frac{v}{c}}^{10}
\end{align*}
 where $\eta\DEF\frac{m_{1}m_{2}}{M^{2}}$. Now we can compute the
energy using 
\[
L_{\text{GW}}=-\frac{\DIFERENTIAL E}{\DIFERENTIAL t}
\]
 and we use the Newtonian approximation for the energy 
\begin{align*}
E & =\frac{m_{2}}{2}v_{1}^{2}+\frac{m_{2}}{2}v_{2}^{2}-\frac{Gm_{1}m_{2}}{a}\\
 & \,\vdots\\
 & =-\frac{\mu}{2}v^{2}
\end{align*}
 Combining these two expressions we get 
\[
\frac{\DIFERENTIAL\PARENTHESES{\frac{v}{c}}}{\DIFERENTIAL t}=\frac{32\eta}{5}\frac{c^{3}}{GM}\PARENTHESES{\frac{v}{c}}^{9}
\]
 Because we are using the Newtonian approximation, this is an approximation...
But in the early moments of the merge we can use this formulas, i.e.
when the two masses are enough far away from each other. 

We can use this result to calculate the \emph{time until coalescence}.
See in the book \cite{Reference: Jolien Gravitational Wave Physics and Astronomy BOOK}
for details. Don't know how accurate this is because the coalescence
happens when the Newtonian approximation is not valid anymore. But
this may happen for a short time.

We can also calculate the phase evolution. Because the gravitational
wave is not monochromatic (because of the decay of the orbital motion)
the phase will not be exactly a linear function increasing with time.
The details of the calculation can be seen in the book. Basically
with this we obtain $\varphi=\varphi\PARENTHESES v$ and in the end
$\cos2\varphi\to\cos2\varphi\PARENTHESES v$. Again, this is Newtonian
so it is an approximation. In the post-Newtonian case the function
$\mathcal{F}$ and $\mathcal{E}$ are replaced by more complicated
post-Newtonian expressions and the calculations are done numerically. 

The post-Newtonian approximations, which are gradually getting better
as the number of terms is increased, are used as ansatz. The merge
of the two bodies is only modeled by full GR, and to get the wave
forms for this you have to numerically solve full GR. But this is
very expensive computationally, so you use post-Newtonian approximations
which are very cheap as ansatz and then you solve the full GR. 

\section*{28.10.2021}

\section{Gravitational wave observation and inference}

What we are discussing today is in hand-written notes, not specifically
taken from anywhere. If it helps, \cite{Reference: Jolien Gravitational Wave Physics and Astronomy BOOK}
chapter 7 has some stuff on this topics. I don't know how much it
is related to this.

\subsection{Classification of data analyses}

We can calssify like this:
\begin{itemize}
\item Utility based\footnote{The following analyses types are in some way sequential, in the sense
that if you fail in answering ``is there a GW here?'' then all the
other analyses cannot be performed, and so on.}:
\begin{enumerate}
\item Detection of the gravitational waves within some data. Basically you
want to say ``yes, there is a GW'' or ``no, there is not''.
\item Source classification and/or parameter estimation. Once you have found
a GW, you want to characterize the source.
\item Others:
\begin{itemize}
\item Test the underlying theory (GR).
\item Test astrophysical formation channels. We use the parameters we have
estimated before in conjunction with a population of events, to test
how were these objects formed. Basically you can test the ``cross
section'' of each type of source based on the rate you observe.
\end{itemize}
\end{enumerate}
\item Source based classification.
\begin{itemize}
\item Transients: You catch a ``short'' transient while your detector
is on for a long time. So $t_{\text{signal}}<t_{\text{observation}}$.
Usually we have 
\[
\LBRACE{\begin{aligned} & t_{\text{observation}}\sim100\text{ to }200\text{ days}\\
 & t_{\text{signal}}\sim1\MILI s\text{ to few days}
\end{aligned}
}
\]
 What we consider $t_{\text{signal}}$ is, of course, the time the
signal is above the noise floor. The sources that produce these signals
are CBC, binaries (BH-BH, NS-NS, BH-NS). Other sources are core collapse,
SN, NS glitches, cosmic strings.
\item Persistent: This is when $t_{\text{signal}}>t_{\text{observation}}$.
Here we can have two types of signals:
\begin{itemize}
\item Continuous wave.
\begin{itemize}
\item Rotating NS. When NS rotate they loose the spherical symmetry, and
this can excite a number of modes that emit GW. The frequency of the
GW is twice the frequency of rotation of the NS. 
\item X-ray binaries.
\item Also NS-NS binaries in an early state. (Also BH-BH and BH-NS, but
NS-NS binaries are known better from the electromagnetic spectrum
so they are more studied.)
\end{itemize}
\item Stochastic background. 
\begin{itemize}
\item Astrophysical background. When you have many many events happening
simultaneously, you see the signals all mixed. As you look farther
in space, the crust of volume at a distance $R$ becomes bigger and
bigger and so it contains more and more events with decreasing amplitude
(because they are further away) but so they mix. So here you have
an ensemble of transient sources.
\item Cosmological background. This is the analogous signal to the CMB,
they are primordial GW. 
\end{itemize}
\end{itemize}
\end{itemize}
\item Methodology based.
\begin{itemize}
\item Un-modeled methods. Rely on the understanding of the detectors. You
take the data and you know your detector, so you try to see if there
is any kind of signal, without worrying too much about which source
produced it. The optimal way of doing this is to try to correlate
the same signal in different detectors.
\item Modeled methods. Rely on the understanding of the source, and you
try to find such signal in your data. The best way of doing this is
to do model-matching to the signals. You can also use machine learning. 
\end{itemize}
\end{itemize}

\subsubsection*{Overhead concept}

Let's recap the response of a GW signal in an interferometer. A source
at position $\theta\phi$ in the sky, and a distance $R$ is producing
the GW with components $h_{+}\PARENTHESES t$ and $h_{\times}\PARENTHESES t$.
So if we have many detectors and we index them with $k$, then 
\[
\EVALUATEDAT{\frac{\Delta L}{L}}k{}=F_{+}\PARENTHESES{\theta,\phi,\psi}h_{+}\PARENTHESES t+F_{\times}\PARENTHESES{\theta,\phi,\psi}h_{\times}\PARENTHESES t
\]
 where $\psi$ I don't know what it is, and the functions $F$ in
principle depend on time because Earth is rotating, but if we look
at short signals (a few seconds) we can neglect this dependence. This
may be explained in more detail in \cite{Reference: Jolien Gravitational Wave Physics and Astronomy BOOK}
� 6.1.10. The response of the detector can be modeled as 
\[
\LBRACE{\begin{aligned} & F_{+}=?\\
 & F_{\times}=\frac{1}{2}\PARENTHESES{1+\cos^{2}\theta}\cos^{2}\phi\sin2\psi+\cos\theta\sin2\phi\cos2\psi
\end{aligned}
}
\]
 Now we define the $\SNR$ squared: 
\begin{align*}
\rho^{2} & \equiv\SNR^{2}\\
 & =4\intop_{0}^{\infty}\frac{\MODULE{\frac{\Delta L}{L}\PARENTHESES f}}{S_{h}\PARENTHESES f}\DIFERENTIAL f
\end{align*}
 where $f$ is the frequency, $\frac{\Delta L}{L}\PARENTHESES f$
is the Fourier transform of $\frac{\Delta L}{L}\PARENTHESES t$ and
$S_{h}\PARENTHESES f$ is the power spectral density. 

\subsection*{Network of detectors }

If we have a network of detectors, the effective $\SNR$ is improved
ass 
\[
\SNR_{\text{effective}}^{2}=\sum_{k\in\text{Detectors}}\SNR_{k}^{2}
\]
 where we are assuming that the detectors are independent in all sense
except for the signal. 

The antenna pattern power is combined like this 
\[
p\PARENTHESES{\theta,\phi}=\sum_{k\in\text{Detectors}}F_{+,k}^{2}+F_{\times,k}^{2}
\]
 where we can define 
\[
p_{k}\PARENTHESES{\theta,\phi}\equiv F_{+,k}^{2}+F_{\times,k}^{2}
\]
 As you add detectors to the network, $p$ gets more uniform. 

In real life the angle between the two arms of an interferometer are
not exactly $90\UNIT{deg}$ because of technological constraints,
geography, etc. So if this angle is $\eta$ then we will have 
\[
\LBRACE{\begin{aligned} & F_{+}=\sin\eta\SQBRACKETS{a\cos^{2}\psi+b\sin^{2}\psi}\\
 & F_{\times}=\sin\eta\SQBRACKETS{a\cos^{2}\psi-b\sin^{2}\psi}
\end{aligned}
}
\]
 where $a$ and $b$ have complicated expressions, see the notes of
the professor. The coordinates are $\beta,\lambda$ the longitude
and latitude of the detector, $\theta\phi$ the position of the source
of GW. 

\subsubsection*{Coincident and coherent formalism}

We have a set of signals $h\PARENTHESES t$ coming from each detector
in the network, e.g. $h^{L}\PARENTHESES t$ and $h^{H}\PARENTHESES t$
from Ligo, $h^{V}\PARENTHESES t$ from Virgo, and so on. How do we
combine all these signals to see if there is something going on? We
can use 
\begin{enumerate}
\item Coincident formalism. Basically you look for time coincidences in
each $h$. We know GW go at the speed of light, we know the distance
between each detector, we can compute a set of time windows withing
the signals should be correlated if the are from the same event. Then
we can look for such signals in all possible time windows in each
$h$. If we find something with a 
\[
\SNR=\sqrt{\sum_{k}\SNR_{k}^{2}}
\]
 bigger than some value, we claim we have an event. 
\item Coherent formalism. Here we not only use time coincidence but also
two other things: The sky direction and the polarization. 
\end{enumerate}

\subsubsection*{Transient detection}

How do we know that we have seen a GW transient? There are glitches
that are of course not GW events but they look like GW events. To
estimate the occurrence of these glitches (e.g. earthquakes, a car
crash, etc) what you do is to shift your data from each detector with
an exagerated $\Delta t$ such that the correlation between GW events
is broken, and then you run your analyses algorithms. In this way
you can estimate how often do glitches that look like GW happen in
your array of detectors. 

So suppose we have two $h$ signals from two detectors. We run our
analysis algorithm and we find an event with some $\SNR_{\text{candidate event}}$.
What is the chance that this was a random glitch and what is the chance
it is an actual GW. What we do to measure the chance a totally random
glitch will produce such an event is to shift one of the $h$ in,
say, $2$ minutes (enough to break correlation of GW events) and then
run the analysis algorithm. Then we repeat this for $3$ minutes,
and for $4$ minutes, and so on. In this way we produce many thousands
of years of backgrand based in real data which we are completely sure
it has no GW events, so here each event we detect with the same $\SNR$
is a fake event. With this we know how often our candidate event can
be produced by a random glitch, say we obtain $10$ glitches in $100000$
years of background, then we would expect to see our $\SNR_{\text{candidate event}}$
once every 10000 years due to random glitches. With this we can compute
the significance of the event for each value of $\SNR$. The lower
the $\SNR$ the lower the significance. With this we can define the
``false alarm probability'' 
\[
\text{FAP}=1-\exp\PARENTHESES{-\frac{t_{\text{observation}}}{i\text{FAR}?}}
\]
 

\section*{04.11.2021}

Unfortunately today I was not able to assist to the lecture.

\section*{11.11.2021}

\subsection*{Recap from the lecture I missed}
\begin{itemize}
\item Transient detection:
\begin{itemize}
\item Unmodeled search (geometric formalism).
\item Modeled search.
\end{itemize}
\end{itemize}

\subsection*{Modeled methods}

For this you need a signal that you can fit with a model. Thus, such
signal must be somehow ``happening all the time''.

To perform a modeled search we need:
\begin{enumerate}
\item Template bank.
\item Likelihood (SNR).
\item Consistency checking (detection statistics).
\end{enumerate}

\subsubsection*{Template bank}

A bank of waveform models. These models should cover all the parameter
space you want to explore (mass, spin, etc). You also need to know
the power spectral density of your detector (to know how probable
is that what you have fitted is a signal or just noise). The parameters
space is discretized to produce the template bank, otherwise it would
be infinite. 
\begin{itemize}
\item Parameter space. If we are looking for signals produced by binaries,
our parameters space is composed by the two masses $m_{1},m_{2}$,
the 3 spin components of each mass $\VECTOR s_{1},\VECTOR s_{2}$,
the eccentricity of the orbit which is one number. There can be other
parameters depending on the model. 
\begin{itemize}
\item The majority of the mergers happen with ``parallel spins'', this
means that $\VECTOR s_{1}\parallel\VECTOR s_{2}\parallel\VECTOR L_{\text{Orbital}}$.
\item Eccentricity is usually 0.
\item The mass can be ``anything''. Usually we consider one of them to
be $m_{\text{high}}$ and the other one to be $m_{\text{low}}$. Typical
values for $m_{\text{high}}\sim100M_{\sun}$ and $m_{\text{low}}\sim1M_{\sun}$. 
\end{itemize}
\end{itemize}
We have to define how we will discretize the parameters space. Suppose
we have two values of the parameters $\theta_{1}$ and $\theta_{2}$
(where $\theta=\BRACES{m_{1},m_{2},\VECTOR s_{1},\VECTOR s_{2},\dots}$).
With these parameters we produce two waveforms $h\PARENTHESES{\theta_{1}}$
and $h\PARENTHESES{\theta_{2}}$. We need a way of measuring the distance
from $h_{1}$ to $h_{2}$ in order to know how to properly discretize
the parameters space. There are many fancy ways of doing this, here
we will define the ``match'' which is 
\begin{align*}
\mu & =\PARENTHESES{\frac{h\PARENTHESES{\theta_{1}}}{\MODULE{\MODULE{h\PARENTHESES{\theta_{1}}}}}|\frac{h\PARENTHESES{\theta_{2}}}{\MODULE{\MODULE{h\PARENTHESES{\theta_{2}}}}}}\\
 & =4\REALPART{\intop_{f_{\text{low}}}^{f_{\text{high}}}\frac{\tilde{h}\PARENTHESES{\theta_{1}}\tilde{h}\PARENTHESES{\theta_{2}}}{\tilde{S}_{n}\PARENTHESES f}\DIFERENTIAL f}
\end{align*}
 where $\tilde{S}_{n}\PARENTHESES f$ is the PSD and $\tilde{h}$
are probably the Fourier transforms. This quantity is measuring how
much SNR you looks if you use $\theta_{2}$ instead of $\theta_{1}$.
For example $\mu=0.9$ tells that you loose $10\UNIT{\%}$ of SNR.

I think this is explained in \cite{Reference: Jolien Gravitational Wave Physics and Astronomy BOOK}
� 7.2.2, or nearby.

\subsubsection*{Likelihood (SNR)}

Look at \cite{Reference: Jolien Gravitational Wave Physics and Astronomy BOOK}
� 7.2.

Our data $d$ is 
\[
d\PARENTHESES t=n\PARENTHESES t+s\PARENTHESES t
\]
 where $n$ is noise and $s$ is signal. We assume that $n$ is Gaussian
with mean 0.

\subsubsection*{Consistency test}

See \cite{Reference: Jolien Gravitational Wave Physics and Astronomy BOOK}
� 7.8.1.4. 

\section*{18.11.2021}

I could not attend this lecture, there is a video \url{https://uzh.zoom.us/rec/share/ZbGfonADZORTd0lpTZZRjfZh1ZbU57f_TsM0qo7_196cRvB-4vfzicYU53NvCn6m._W6fY85VrZG1gSql}.

\section*{25.11.2021}

When we use get\_td\_waveform and functions, we have to be careful
that 
\[
\LBRACE{\begin{aligned} & \frac{\text{mass 1}}{\text{mass 2}}<20\\
 & \frac{\text{mass 2}}{\text{mass 1}}<20
\end{aligned}
}
\]
 because otherwise it is not valid anymore.

\section*{02.12.2021}

\section{Source modeling 2}

Today we are going to finish some stuff that was pending from before. 

\begin{figure}[H]
\begin{centering}
\htmltag{tag_name=image}{src=media/3.svg}{style=max-width: 100\%;}
\par\end{centering}
\caption{Gravitational wave}
\end{figure}

Each stage is calculated in a different way, the most complicated
is the plunge and merger stages where numerical relativity has to
be used. 

\subsection{Numerical relativity}

In 1952 it was probed that the EFE in vacuum are well posed, i.e.
that there is solution and that small changes in initial conditions
produce small changes in solutions so this is stable and continuous. 

The EFE are 10 quasi-linear wave-like equations with hyperbolic stability
properties. 

In 1962, 1972 it was introduced the ``ADM formalism'' which looks
into the Cauchy problem for PDEs and then formulates the EFE as a
computable and iterable process in time steps that go from the initial
state to the final state. 

\subsubsection*{ADM formalism}

3+1 decompositions (3-dim space like slices with fixed coordinate
times. 

\begin{figure}[H]
\begin{centering}
\htmltag{tag_name=image}{src=https://www.frontiersin.org/files/Articles/536296/fspas-07-00058-HTML/image_m/fspas-07-00058-g001.jpg}{style=max-width: 100\%;}
\par\end{centering}
\caption{ADM formalism.}
\end{figure}

In 4 dimensions the interval is, as we know, $\DIFERENTIAL s^{2}=g\COVARIANTINDEX{\mu\nu}\DIFERENTIAL x\CONTRAVARIANTINDEX{\mu}\DIFERENTIAL x\CONTRAVARIANTINDEX{\nu}$.
In 3+1 formalism this is written as 
\[
\DIFERENTIAL s^{2}=-\alpha^{2}\DIFERENTIAL t^{2}+\gamma\COVARIANTINDEX{ij}\PARENTHESES{\DIFERENTIAL x\CONTRAVARIANTINDEX i+\beta\CONTRAVARIANTINDEX i\DIFERENTIAL t}\PARENTHESES{\DIFERENTIAL x\CONTRAVARIANTINDEX j+\beta\CONTRAVARIANTINDEX i\DIFERENTIAL t}
\]
 where $\gamma\CONTRAVARIANTINDEX{ij}$ are 3 dimensions metrics and
$\alpha,\beta\COVARIANTINDEX i$ are gauge variables. $\alpha$ is
called the ``lapse'' and measures the rate in proper time progression
between two slices of spacetime while $\beta\COVARIANTINDEX i$ is
sthe shift vector and measures the change of spacial coordinates between
different slices. See the previous figure. The lapse $\alpha$ has
to be chosen in order to avoid singularities, improve convergence
and speed up the simulation.

There is also an ``extrinsic curvature tensor'' $K\COVARIANTINDEX{ij}$
that separates the intrinsic curvature of $\gamma\COVARIANTINDEX{ij}$
from the extrinsic one. 

In the EFE we have 6 evolution equations + 4 constraint equations.
In the ADM formalism look \href{https://en.wikipedia.org/wiki/ADM_formalism}{here},
there are 3 equations of motion for $\gamma\COVARIANTINDEX{ij}$ and
3 for $K\COVARIANTINDEX{ij}$ which come from the 6 evolution equations
in EFE, and the 4 constraint equations are applied to $\gamma\COVARIANTINDEX{ij}$
and $K\COVARIANTINDEX{ij}$. 

With this formalism there are two types of equations:
\begin{enumerate}
\item Hyperbolic, i.e. wave-like equations. These are nice because they
have a rapid convergence.
\item Parabolic, i.e. time-dependent. 
\end{enumerate}
These problems are mathematically stable once initial constraints
are satisfied. However in full GR there is a lot of non-linearity
and the small numerical computational errors can blow up. This blow
up due to the non-linearity is the challenge of numerical relativity.
For this we have to keep the constraints satisfied at all times and
this guaranties that stable and convergent solutions exist. 

\subsection{Suitable initial conditions}

\subsubsection{BSSNOK approach}

There is one problem which is to propose suitable initial conditions.
This was solved during the 1990s by Baugmarde, Shapiro (1998) and
Shibala, Nakamura (1995) in the so called BSSNOK approach. The problem
is that we have 12 real numbers that are the initial data for $\gamma\COVARIANTINDEX{ij}$
and $K\COVARIANTINDEX{ij}$. These numbers are not all independent,
they are related to the parameters of the system, e.g. the mass and
spin of the black holes, and the initial orbit dynamics. We want to
know how are they related so we can solve the problem. 

\subsubsection{Generalized harmonic coordinates with constraint damping (GHCD)}

This is a different approach. Not understanding too much. There are
two unstable scalar field profiles: initial amplitudes, boosts, separaitons.
These are chosen such that approximates two black holes in orbit. 

\subsection{Handling of physical singularities of black holes}

There is another problem. 

\subsubsection{Excision}

This is one way of solving the singularities problem. We have different
types of black holes, Schwarzschild and Kerr. Both have event horizons
that shield the singularities. What we do is to simulate outside the
event horizon if I understood correctly. (Unruh 1984)

There are challenging boundary conditions due to irregular shape of
black hole horizons. This creates numerical artifacts and in particular
emission of gravitational waves that is unphysical. To mitigate this
problem you have to fine tune the constraints. 

\subsubsection{Punctures}

This is another way of solving the singularities problem. This solution
was introduced by Hahn, Lindquist in 1964 axisymmetric coalescence
of two black holes. Here you somehow design your space time such that
the solution avoids the singularities using wormholes, or something
similar. 

This technique lead to the first successful simulation of two black
holes, not until 2005. Some modifications to the original punctures
approach were done, for example letting the punctures to freely move
in the coordinates system. 

\subsection{Discretization and gridding}

This is another problem. In the initial problem, probably after a
Newtonian ansatz, there are two BH well separated in a flat spacetime.
Then you evolve the simulation and you go into coalescence where the
two black holes are close and the space time is highly curved. The
two situations have very different length scales so the numerical
mesh that we take in each case has to be very different. We need an
``adaptive mesh refinement''. In this field not much progress was
done.

\subsection{Integration procedure}

This is another problem. There are different options
\begin{itemize}
\item Finite difference methods (FD) which approximate the solution of the
PDEs at specific points in the mesh.
\item Spectral interpolation methods (SpEC). Fit smooth functions to several
points in the mesh. This produces accurate solutions at any location
in the mesh. 
\end{itemize}

\subsection{Gravitational wave extraction}

Another problem. I don't know what do we want to do now. They use
the Newman-Penrose approach. They calculate the Weyl scalar $\psi_{4}$
approximation of GW radiation at spatial infinity. 

\subsection{Phenomenological IMR modeling in frequency domain}

These are very fast because they are analytical, you don't need to
solve any equation. It is done in frequency domain because of convenience.
They combine analytical approaches from numerical relativity. To go
to the frequency domain we use the ``stationary phase approximation''

\section*{09.12.2021}

\section{Compact binaries}

Compact binaries provide a good system to study 
\begin{itemize}
\item Astrophysics.
\item Test of GR.
\item Cosmology.
\item Extreme matter.
\end{itemize}
We will refer to the \href{https://en.wikipedia.org/wiki/Main_sequence}{main sequence}
as the ``ZAMS'' which comes from ``zero age main sequence''. 
\begin{figure}[H]
\begin{centering}
\htmltag{tag_name=image}{src=https://upload.wikimedia.org/wikipedia/commons/thumb/6/6b/HRDiagram.png/465px-HRDiagram.png}{style=max-height: 90vh;}
\par\end{centering}
\caption{ZAMS. }
\end{figure}


\section*{16.12.2021}

\section{Tests of GR}

I arrived some minutes late today.

\subsection{Strong-field tests of GR}

Here we have binary pulsars. In 1974 it was found the first binary
pulsar \href{https://en.wikipedia.org/wiki/Hulse\%E2\%80\%93Taylor_binary}{PSR B1913+16}.
Because these are very dense objects, the system emits gravitational
waves. This loss of energy reduces the orbital period, and this was
measured to be in agreement with GR. This system is in a very early
stage of the merge. 

\begin{figure}[H]
\begin{centering}
\htmltag{tag_name=image}{src=https://upload.wikimedia.org/wikipedia/commons/thumb/0/04/PSR_B1913\%2B16_period_shift_graph.svg/800px-PSR_B1913\%2B16_period_shift_graph.svg.png}{style=max-height: 90vh;}
\par\end{centering}
\caption{Orbital decay of PSR B1913+16.}
\end{figure}

In 2003 a \href{https://en.wikipedia.org/wiki/PSR_J0737\%E2\%88\%923039}{double pulsar system}
was discovered. This also provides a good test for GR and up to now
is in agreement. 

\subsection{Tests of GR with compact binary coalescence}

We can use the gravitational wave events from LIGO and company to
test GR.
\begin{itemize}
\item We can perform a ``residuals test'' which is basically to look at
$\text{data}-\text{model}$. If the model is good, you will only see
noise here, if there is something wrong in your model, you will see
not just noise but a component of the signal that is not in the model. 
\item We can also do a ``least-damped QNM'' which looks if the data is
consistent with GR prediction of QNMs (quasi normal modes) of the
remnant black hole. For this there is a signal model for $t>t_{0}$
where $t_{0}$ is the time at which the QNM start, and loos like $h\PARENTHESES t\sim e^{-\frac{t-t_{0}}{\tau}}\cos\PARENTHESES{2\pi f_{0}\PARENTHESES{t-t_{0}}+\phi_{0}}$. 
\item Waveform self-consistengy. There is a prediction for the waveform
and it has to be consistent. This means that if we estimate the final
mass and spin of the black hole after the merger in two ``independent
ways''
\begin{enumerate}
\item using data from the inspiral stage and
\item using data from the ringdown + merger stage
\end{enumerate}
we want the mass and spin to be the same, independently of which of
these regions we use. 
\item Massive graviton? If the graviton is massive, there should be dispersion
in the propagation of gravitational waves. Up to now there is on evidence
for dispersion and $m_{\text{graviton}}\le1\TIMESTENTOTHE{-22}\UNIT{eV}$. 
\item Parameterized deviations from GR. This I think it is basically to
see if other things that are not exactly GR fit better the data than
GR does.
\item Difference in the speed of light and gravity. This is done combining
data from gravitational wave and electromagnetic observations of the
same event, and then we can compare the time of arrival of each. 
\end{itemize}

\end{document}
